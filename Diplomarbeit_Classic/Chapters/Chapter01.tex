%************************************************
\chapter{Introduction}\label{ch:introduction}
%************************************************
\section{Introduction}
Navigation and planning are essential for mobile robots to act in out- and indoor environments. 
From self driving cars navigating 132~miles through the Mojave desert, or mobile robots handling goods in distribution centers and warehouses to mobile robots used for planetary exploration, autonomous robots have conquered every place on earth and beyond.
 
Articles like \cite{stanley} and \cite{kiva} present the relationship between the environmental complexity and the computational on-board power needed to deal with it.
Stanley has a six processor computing platform sponsored by Intel whilst Kiva robots are using low cost DSP's for navigation and vision processing\footnote{Kiva Systems Uses "Smart" Blackfin-powered Robots for Warehouse Navigation | Analog Devices: \url{http://www.analog.com/en/content/kiva_systems_bf548/fca.html}} to drive within a known environment.  

While the application domains and computational powers varies strongly between the robotic systems, all of them have to move safely and efficient from one location to another.   
In order to give an impression of the importance of navigation and planning in the field of mobile robotics the next section gives some motivation and application examples.

\section{Motivation and Applications}\label{sec:motivation} 
The goal of navigation encompasses the ability of robots to find a series of actions based on its knowledge of the environment and sensor values to reach its goal position in an effective and efficient manner.
The resulting series of actions is called a \emph{plan}. 
To ensure safety and flexibility in the presence of obstacles in a dynamic environment \emph{obstacle avoidance} is used to alter plans during execution and generating of collision free trajectories.
A common strategy to deal with complex planning problems is to the divide the navigation task into a global and a local planning problem \cite{LaValle2006}.
Global path-planning requires a simplified representation, e.g. static map, of the search problem to efficiently compute an optimal shortest path using variants of Dijkstra's \cite{dijkstra1959note} or $A^*$ \cite{DBLP:journals/tssc/HartNR68/Astar} algorithm, ignoring kinematic and acceleration constraints of the robot.
In succession the retrieved global path is used by a local planner for guiding the robot through the environment.
Figure~\ref{fig:fig_pioneer} illustrates the view of the environment from a robot perspective together with a global and local plan, which enables the robot to drive autonomous within our lab/office environment.

\begin{figure}[thpb]
      \centering
      \def\svgwidth{0.7\textwidth}
      \includesvg{figures/pioneer_costmap}
      \caption{This figure shows a Pionner3DX and its view of the office environment while passing through a door. The blue line shows the global path and the green line the selected trajectory of the local planner.}
      \label{fig:fig_pioneer}
\end{figure}

The major responsibility for local planner is obstacle avoidance. It takes sensor readings of the robot into account and is reactive to changes within the sensor range. 
It selects the best values of available motor controls in respect to the kinematic and dynamic constraints of the robot, generating collision free trajectories. 
   
Navigation competence is essential for a broad spectrum of application domains within the field of mobile robotics which are presented below:

\begin{description}
\item[Autonomous Vehicles]

Stanley \citep{stanley}
DARPA Challenge Courtesy of Carnegie Mellon University

Google Cars \url{http://googleblog.blogspot.co.at/2014/05/just-press-go-designing-self-driving.html}.

Planetary Rovers
three generations of Mars rovers developed at NASA
Front and center is the flight spare for the first Mars rover, Sojourner, which landed on Mars in 1997 as part of the Mars Pathfinder Project. On the left is a Mars Exploration Rover Project test rover that is a working sibling to Spirit and Opportunity, which landed on Mars in 2004. On the right is a Mars Science Laboratory test rover the size of that project's Mars rover, Curiosity, which is on course for landing on Mars in August 2012.
Image credit: NASA/JPL-Caltech 
  
\item[Search and Rescue]
Taurob
PIONEER Tschernobyl
CMU Robotics Institute, the first of its kind in the world, provides the robot Pioneer to the Ukraine to evaluate the sarcophagus around the damaged nuclear power station at Chernobyl. 
PACKBOT Fukushima
iRobot recently deployed a pair of robots to the Fukushima Daiichi nuclear plant in Japan, where intense levels of radiation have made it increasingly dangerous for human rescue workers to operate.
Search and Rescue winner Robocup 2014

\item[Logistics and Transportation]
AGV DS-Automation
HELPMATE
Kiva Systems \cite{kiva}
UAV Project Wing

\item[Assistance and Companion Robots]
Companion KIBA, 
Health Care HOBBIT, 
Guide RHINO, ROBOX
Strands Henry

\item[Commercial]
Mining Robots, law mowing, floor cleaning, harvesting
FRANC and ARW Paper
inpipe robot 

\item[Challenges and Competitions]
FIRA
Robocup,Robotsoccer Standard platform League, other Leagues
FIRA3 and RoboCup 4 are two organizations intended to promote robot soccer. They
attempt to provide a standard problem where wide ranges of technologies must be combined
and integrated.
Federation of International Robot-soccer Association, founded in 1997. Details can be found at www.fira.net
RoboCup Federation, founded in 1993. Details can be found at www.robocup.org.

Robot@Home
RobocupLogisticLeague
DARPA

\end{description}

Other fields not in robotic environment. Knot entangling puzzles, automation and assembly, Molecular biology and medicine.
And a short summary towards the goal of the thesis.

\section{Goal of Thesis and Scientific Contribution}\label{sec:goal}


One of the most popular local planer and reactive collision avoidance method is the Dynamic Window Approach (DWA)\cite{DWA1997}. Its based on evaluating a fixed number of trajectory samples in a reduced velocity space. 
A dynamic window around the current robot fused with the current sensor readings of the robot represents a so called costmap.
Trajectories are sampled into that costmap and weighted by a cost function.
Recent adoptions of this method can be found in \cite{conf/icra/SederP07}\cite{DBLP:conf/icra/Marder-EppsteinBFGK10}

In this paper we propose a method which finds the best velocity commands by using search strategies based on Meta-Heuristics instead of evaluating a fixed number of trajectories in a brute force manner.
A combination of Iterated Local Search (ILS) with different neighborhood structures, Variable Neighborhood Search (VNS), and a Tabu list provides a significant documented performance boost. 

This enables the local planer to:
\begin{itemize}
\item run at a higher frequency
\item simulate trajectories for a longer time interval
\item moving the robot at higher speed
\item investigate a larger amount of trajectories
\item use a higher costmap resolution
\end{itemize}

\section{Outline}\label{sec:outline}
This work is structured in the following way. 
In Chapter \ref{ch:introductionplanning} the basic sources for planning are outlined, covering the basic problem definition, representation of the environment and different robotic motion models. 

Chapter \ref{ch:planningalgorithms} gives an overview of important planning algorithms and obstacle avoidance methods.
 
Chapter \ref{ch:meta} represents the main part of this thesis and introduces meta-heuristic extensions to local planning methods. 
Starting with an detailed overview of  meta-heuristic algorithm, two application based on ILS and VNS are used for trajectory selection of a local planner and described in Section \ref{sec:trajselmeta}.

A detailed evaluation of these methods is presented in Chapter \ref{ch:eval}, using a very simple implementation of a local planner and randomly generated sensor maps. 
The gathered data is then used to implement a meta-heuristic trajectory selection in a high sophisticated planner and tested using simulation software with a robust physical engine.

Chapter \ref{ch:conc} concludes the work of this thesis with a short summary and an outlook on future research directions.
%*****************************************
%*****************************************
%*****************************************
%*****************************************
%*****************************************




