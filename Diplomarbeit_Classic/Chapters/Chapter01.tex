%************************************************
\chapter{Introduction}\label{ch:introduction}
%************************************************
\section{Introduction}
Navigation and planning are essential for mobile robots to act in out- and indoor environments. 
From self driving cars navigating 132~miles through the Mojave desert, or mobile robots handling goods in distribution centers and warehouses to mobile robots used for planetary exploration, autonomous robots have conquered every place on earth and beyond.
 
Articles like \cite{stanley} and \cite{kiva} present the relationship between the environmental complexity and the computational on-board power needed to deal with it.
Stanley has a six processor computing platform sponsored by Intel whilst Kiva robots are using low cost DSP's for navigation and vision processing\footnote{Kiva Systems Uses "Smart" Blackfin-powered Robots for Warehouse Navigation | Analog Devices: \url{http://www.analog.com/en/content/kiva_systems_bf548/fca.html}} to drive within a known environment.  
While the application domain and computational power varies strongly between the robotic systems, all of them have to move safely and efficient from one location to another.   
In order to give an impression of the importance of navigation and planning in the field of mobile robotics the next section gives some motivation and application examples.

\section{Motivation and Applications}\label{sec:motivation} 

\begin{description}
\item[Autonomous Vehicles]
Stanley \citep{stanley}
Google Cars
Planetary Rovers
UAV Project Wing
\item[Search and Rescue]
Taurob
PIONEER Tschernobyl
\item[Logistic and Transportation]
AGV DS-Automation
HELPMATE
Kiva Systems \cite{kiva}
\item[Assistance and Companion Robots]
Companion KIBA, 
Health Care HOBBIT, 
Guide RHINO, ROBOX
\item[Challenges and Competitions]
FIRA
Robocup
Robotsoccer Standard platform League, other Leagues
Robot@Home
DARPA
\item[Commercial]
Mining Robots, law mowing, floor cleaning, harvesting
\end{description}

Other fields not in robotic environment. Knot entangling puzzles, automation and assembly, Molecular biology and medicine.
And a short summary towards the goal of the thesis.

\section{Goal of Thesis and Scientific Contribution}\label{sec:goal}
A common strategy to deal with the complex planning problem is the division into a global and a local planning problem \cite{LaValle2006}.
Global path-planning requires a simplified representation, e.g. static map, of the search problem to efficiently compute an optimal shortest path using variants of Dijkstra's \cite{dijkstra1959note} or $A^*$ \cite{DBLP:journals/tssc/HartNR68/Astar} algorithm, ignoring kinematic and acceleration constraints of the robot.
In succession the retrieved global path is used by a local planner for guiding the robot through the environment.
Figure~\ref{fig:fig_pioneer} illustrates the view of the environment from a robot perspective together with a global and local plan, which enables the robot to drive autonomous within our lab/office environment.

\begin{figure}[thpb]
      \centering
      \def\svgwidth{0.7\textwidth}
      \includesvg{figures/pioneer_costmap}
      \caption{This figure shows a Pionner3DX and its view of the office environment while passing through a door. The blue line shows the global path and the green line the selected trajectory of the local planner.}
      \label{fig:fig_pioneer}
\end{figure}

%(REMARK: global path planning in la valle, D*, FD* usw. eventuel)%
The local planner takes sensor readings of the robot into account and is reactive to changes within the sensor range. 
It chooses the best values of available motor controls in respect to the kinematic and dynamic constraints of the robot. 
The main task is to avoid collision with obstacles, by generating feasible velocity commands to produce a trajectory for the robot near the global path.
The heuristic strategy presented optimizes the local planner.   

One of the most popular local planer and reactive collision avoidance method is the Dynamic Window Approach (DWA)\cite{DWA1997}. Its based on evaluating a fixed number of trajectory samples in a reduced velocity space. 
A dynamic window around the current robot fused with the current sensor readings of the robot represents a so called costmap.
Trajectories are sampled into that costmap and weighted by a cost function.
Recent adoptions of this method can be found in \cite{conf/icra/SederP07}\cite{DBLP:conf/icra/Marder-EppsteinBFGK10}
%(REMARK: Other methods with fixed sized samples: Curvatur Velocity Method. Lane curvature method, lookup sources, cite resent papers which use this apporaches)

In this paper we propose a method which finds the best velocity commands by using search strategies based on Meta-Heuristics instead of evaluating a fixed number of trajectories in a brute force manner.
A combination of Iterated Local Search (ILS) with different neighborhood structures, Variable Neighborhood Search (VNS), and a Tabu list provides a significant documented performance boost. 

This enables the local planer to:
\begin{itemize}
\item run at a higher frequency
\item simulate trajectories for a longer time interval
\item moving the robot at higher speed
\item investigate a larger amount of trajectories
\item use a higher costmap resolution
\end{itemize}

\section{Outline}\label{sec:outline}
Of particular interest are local planning methods which apply sampling and simulation of trajectories. 
Here the maximization step can easily be substituted by the proposed method. 

%*****************************************
%*****************************************
%*****************************************
%*****************************************
%*****************************************




