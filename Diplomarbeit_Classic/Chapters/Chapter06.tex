% Chapter 6
\chapter{Conclusions}\label{ch:conc}
In this work a promising approach to improve performance of existing local planning systems which are based on generate and test trajectory methods by using meta-heuristic search strategies was analyzed.

To this end this thesis presented a thorough overview of planning methods with the focus on local planning methods and obstacle avoidance. 
Identifying trajectory selection as the main part of local planning tasks for improvement, well known meta-heuristic search algorithms were introduced for optimizing the selection process.

From the large family of meta-heuristic algorithms the following search strategies were selected to substitute the brute force approach of trajectory selection in a \emph{DWA} based local planning method:
\begin{itemize}
\item The basic algorithmic structure of the single solution based meta-heuristics \emph{Iterated Local Search (ILS)} and \emph{Variable Neighborhood Search (VNS)}.
\item An appropriate formulation of a set of \emph{neighborhood} structures, which are used during \emph{Local Search}  with \emph{best improvement} and \emph{first improvement} heuristic.
\item Two basic \emph{memory structures} inspired from \emph{tabu search} to avoid infeasible solutions and short cycles during the search process. 
\end{itemize}

To evaluate the proposed method a set of random generated environments simulating sensor information for one planning step were created. 
These instances were used to test the algorithms using an example planner. 
This allowed a focused investigation and improvement of the trajectory selection part separated from the overhead of full planning systems.

The results of these experiments with the example planner showed, that the meta-heuristic algorithms provide significant performance improvement on all of the test instances. Especially the VNS implementations provided very good and stable results. 

In a next step the VNS approach was used to extent a popular existing planner implementation and tested using a sophisticated simulation environment with realistic scenarios and physics. Two configurations were tested in an virtual building:
\begin{itemize}
\item VNS-ROL: Using VNS with best improvement heuristic to extend the roll-out method of the planner with continues acceleration limits.
\item VNS-DWA: Using VNS with best improvement heuristic to extend the classical DWA method of the planner.
\end{itemize}
Both extensions showed superior run time performance compared to their unaltered counterparts.

Summarizing the meta-heuristic algorithms are able to increase the performance of local planning systems based on trajectory generate and test methods. 
This allows the local planner to improve on \emph{reactivity}, the used \emph{resolution} for collision tests, the \emph{simulation-time} for look-ahead and the number of \emph{trajectories} used to model the robots motion capabilities.

\section{Further Research}
Applying meta-heuristic search to trajectory selection of local planners like DWA shows to be a promising step in using the power of these search procedures in the context of local planning. 
Therefore the following directions for further research are suggested:

\begin{description}
\item[Investigate additional meta-heuristics]\hfill \\
This work presented the use of a very limited selection of single solution based meta-heuristics. Therefore it would be of interest to investigate related algorithm like GRASP \cite{feo1995grasp}, reduced VNS, or Simulated Annealing \cite{Kirkpatrick83SimulatedAnnealing} or even population based meta-heuristics.
In addition, developing and analyzing more sophisticated neighborhood structures would be recommended. 

\item[Application to high DOF robotic models]\hfill \\
The proposed method might also be applicable for robot models of higher degree of freedom, since dealing with large trajectory samples is a particular strength of meta-heuristic search.

\item[Extending to three-dimensional space]\hfill \\
This work considers exclusively two-dimensional workspaces. Application of this method to three-dimensional spaces is an interesting challenge and would allow the use of new obstacle avoidance methods for robotic models like unmanned air vehicles (UAV).

\item[Extending other planning systems]\hfill \\
The applicability to similar path-planning methods using trajectory samples, like Curvature Velocity Method \cite{simmons1996curvature}, would be of interest to show the general applicability of this method. 


\item[Increasing the planning horizon]\hfill \\
The extended planning methods simulate trajectories by applying a constant amount of velocity over a given time. This restriction simplifies the planning step significantly and allows for brute force evaluation of trajectories. With the help of the proposed method the evaluation of chaining short series of different velocity commands may still be feasible and would allow for a better lookahead simulation, allowing more sophisticated maneuver.

\end{description}
